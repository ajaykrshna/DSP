\documentclass{article}
\usepackage{listings}
\usepackage[margin=2cm]{geometry}


\title{DSP LAB - LAB 1}
\author{Ajay Krishnan K - EE22BTECH11003}
\date{\today}

\begin{document}

\maketitle

\section{Convolution}
\subsection{MATLAB}
\subsubsection{Code}
\begin{lstlisting}[language=Matlab]
    % CONVOLUTION

    clc
    clear
    close all
    
    a = input('Enter the signal: ');
    b = input('Enter the impulse response: ');
    
    % a = [1 2 3];
    % b = [4 5 6];
    conv_val = conv_taker(a,b);
    disp(conv_val)
    
    function [conv] = conv_taker(signal, impulse)
    L = length(signal) + length(impulse) - 1;
    conv = zeros(1, L);
    
    for n = 1:L
        for k = 1:length(impulse)
            m = n - k + 1;
            if 1 <= m && m <= length(signal)
                conv(n) = conv(n) + signal(m) * 
                impulse(k);
            end
        end
    end
    end
    
\end{lstlisting}

\subsubsection{Output}
\begin{lstlisting}[language=Matlab]
    Enter the signal: [0.3426 3.5784 2.7694 -1.3499 3.0349 0.7254 -0.0631]
    Enter the impulse response: [0.7147 -0.2050 -0.1241 1.4897 1.4090]
        0.2449    2.4872    1.2032   -1.4662    7.9156    
            9.2314    1.3207    2.5420    5.3646    0.9281   -0.0889
\end{lstlisting}

\subsection{C}
\subsubsection{Code}
\begin{lstlisting}[language=C]
    #include <stdio.h>

    void convolve(int signal[], int impulse[],int signal_length,int impulse_length,
      int result[]) {
    int i, j, k;

    for (i = 0; i < signal_length + impulse_length - 1;
     i++) {
        result[i] = 0;
        for (j = 0; j <= i && j < signal_length; j++) {
            k = i - j;
            if (k < impulse_length) {
                result[i] += signal[j] * impulse[k];
                }
            }
        }
    }

    int main() {
    int signal_length, impulse_length, i;
    printf("Enter the size of the first array: ");
    scanf("%d", &signal_length);
    int signal[signal_length];
    printf("Enter the elements of the first array: ");
    for (i = 0; i < signal_length; i++) {
        scanf("%d", &signal[i]);
    }

    printf("Enter the size of the second array: ");
    scanf("%d", &impulse_length);
    int impulse[impulse_length];
    printf("Enter the elements of the second array: ");
    for (i = 0; i < impulse_length; i++) {
        scanf("%d", &impulse[i]);
    }

    int result[signal_length + impulse_length - 1];
    convolve(signal, impulse, signal_length, impulse_length,result);

    printf("The convolution of the two arrays is: ");
    for (i = 0; i < signal_length + impulse_length - 1;i++) {
        printf("%d ", result[i]);
    }
    printf("\n");

    return 0;
    }

\end{lstlisting}

\subsubsection{Output}
\begin{lstlisting}[language=Bash]
    Enter the size of the first array: 7
    Enter the elements of the first array: 0.3426 3.5784 2.7694 -1.3499 3.0349 0.7254 -0.0631
    Enter the size of the second array: 5
    Enter the elements of the second array: 0.7147 -0.2050 -0.1241 1.4897 1.4090
    The convolution of the two arrays is: 0.245 2.487 1.203 
    -1.466 7.916 9.231 1.321 2.542 5.365 0.928 -0.089
\end{lstlisting}

\section{Correlation}

\subsection{MATLAB}
\subsubsection{Code}
\begin{lstlisting}
    % CORRELATION

clc
clear
close all

% a = input('Enter the signal: ');
% b = input('Enter the impulse response: ');

a = [1 2 3];
b = [4 5 6];

correlation_val = correlation_taker(a,b);
disp(correlation_val)

function [correlation] = correlation_taker(signal, impulse)
L = length(signal) + length(impulse) - 1;
correlation = zeros(1, L);

for n = 1:L
    for k = 1:length(impulse)
        m = n - k + 1;
        if 1 <= m && m <= length(signal)
            correlation(n) = correlation(n) + signal(m) * impulse(length(impulse) - k + 1);
        end
    end
end
end
\end{lstlisting}

\subsubsection{Output}
\begin{lstlisting}
    
    Enter the signal: [0.3426 3.5784 2.7694 -1.3499 3.0349 0.7254 -0.0631]
    Enter the seconfd signal: [0.7147 -0.2050 -0.1241 1.4897 1.4090]
        0.4827    5.5523    9.1903    1.7093    1.4328    
            7.7005    2.8711   -1.7710    2.0282    0.5314   -0.0451

\end{lstlisting}


\subsection{C}
\subsubsection{Code}
\begin{lstlisting}
    #include <stdio.h>

    void correlate(float signal[], float impulse[], 
    int signal_length, int impulse_length, float result[])
    {
        int i, j, k;

        for (i = 0; i < signal_length + impulse_length - 1; i++)
        {
            result[i] = 0;
            for (j = 0; j <= i && j < signal_length; j++)
            {
                k = i - j;
                if (k < impulse_length)
                {
                    result[i] += signal[j] * impulse[impulse_length - 1 - k];
                }
            }
        }
    }

    int main()
    {
        int signal_length, impulse_length, i;
        printf("Enter the size of the first array: ");
        scanf("%d", &signal_length);
        float signal[signal_length];
        printf("Enter the elements of the first array: ");
        for (i = 0; i < signal_length; i++)
        {
            scanf("%f", &signal[i]);
        }

        printf("Enter the size of the second array: ");
        scanf("%d", &impulse_length);
        float impulse[impulse_length];
        printf("Enter the elements of the second array: ");
        for (i = 0; i < impulse_length; i++)
        {
            scanf("%f", &impulse[i]);
        }

        float result[signal_length + impulse_length - 1];
        correlate(signal, impulse, signal_length, impulse_length, result);

        printf("The correlation of the two arrays is: ");
        for (i = 0; i < signal_length + impulse_length - 1; i++)
        {
            printf("%0.3f ", result[i]);
        }
        printf("\n");

        return 0;
    }

\end{lstlisting}

\subsubsection{Output}
\begin{lstlisting}
    Enter the size of the first array: 7
    Enter the elements of the first array: 0.3426 3.5784 2.7694 -1.3499 3.0349 
    0.7254 -0.0631
    Enter the size of the second array: 5
    Enter the elements of the second array: 0.7147 -0.2050 -0.1241 1.4897 1.4090
    The correlation of the two arrays is: 0.483 5.552 9.190 1.709 1.433 7.700
     2.871 -1.771 2.028 0.531 -0.045 
\end{lstlisting}



\end{document}
