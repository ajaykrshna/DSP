\documentclass[a4paper,12pt]{article}

% Packages
\usepackage[utf8]{inputenc}
\usepackage{geometry}
\usepackage{titlesec}
\usepackage{graphicx}
\usepackage{caption}
\usepackage{subcaption}
\usepackage{listings}
\usepackage{amsmath}
\usepackage{amssymb}                                                                                                                                                            
\usepackage{xcolor}

% Page setup
\geometry{a4paper, margin=1in}
\setlength{\parindent}{0pt}
\setlength{\parskip}{5pt}

% Title setup
\title{\textbf{DSP LAB - Experiment 8} \\
        \vspace*{0.3em}
        \large{Interpolation and Decimation} \\}                                              
\author{Ajay Krishnan K \\  EE22BTECH11003}
\date{\today}

% Section and subsection formatting
\titleformat{\section}[block]{\normalfont\Large\bfseries}{\thesection}{1em}{}
\titleformat{\subsection}[block]{\normalfont\large\bfseries}{\thesubsection}{1em}{}
\titleformat{\subsubsection}[block]{\normalfont\normalsize\bfseries}{\thesubsubsection}{1em}{}

% Code listing settings
\lstdefinestyle{mystyle}{
    language=Matlab,
    basicstyle=\ttfamily\small,
    breaklines=true,
    keywordstyle=\color{blue},
    commentstyle=\color{green!40!black},
    stringstyle=\color{red},
    % numbers=left,
    % numberstyle=\tiny,
    frame=single,
    showspaces=false,
    showstringspaces=false,
}

\lstset{style=mystyle}

\begin{document}
\maketitle

% o Aim of the experiment
% o Theory of Interpolator and Decimator.
% o How filter cutoff frequencies are decided in
% Interpolation and decimation.
% o Matlab code and mean of absolute error.
% o Observation and conclusion.

\section*{Aim}
To implement interpolation and decimation of a given signal and 
calculate the mean of absolute error between the original signal and the decimated signal.

\section*{Theory}
\subsection*{Interpolation}
Interpolation is the process of increasing the sampling rate of a signal. 
This is done by inserting zeros between the samples of the signal and then applying a 
low-pass filter to remove the high frequency components introduced by the zero padding.

\subsection*{Decimation}
Decimation is the process of decreasing the sampling rate of a signal.
This is done by initially passing the signal through a low-pass filter to 
remove the high frequency components and then removing samples from the signal.

\section*{Procedure}
\begin{enumerate}
    \item Generate a signal $x[n] = \sin(2\pi f_1 n) + \sin(2\pi f_2 n) + \sin(2\pi f_3 n)$ where $f_1 = 500$, $f_2 = 1000$ and $f_3 = 700$.
    \item Interpolate the signal by a factor of $L = 2$.
    \item Decimate the interpolated signal by a factor of $M = 2$.
    \item Observe the effect of interpolation and decimation on the signal.
\end{enumerate}

\section*{Design}
\subsection*{Interpolation}
The interpolation of a signal is done by inserting zeros between the samples of the signal.
The interpolated signal is then passed through a low-pass filter to remove the high frequency components introduced by the zero padding.

The cutoff frequency of the low-pass filter is given by
\begin{align*}
    w_c = \frac{\pi}{L}
\end{align*}
where $f_s$ is the sampling frequency of the signal and $L$ is the interpolation factor.

Since the signal is being interpolated by a factor of L, the sampling frequency of the signal is also increased by a factor of L.
Thus, the equation.

A gain of L is applied to the output of the low-pass filter to compensate for the loss of energy due to the zero padding.

\subsection*{Decimation}
The decimation of a signal is done by removing samples from the signal.
The decimated signal is then passed through a low-pass filter to remove the high frequency components introduced by the decimation.

The cutoff frequency of the low-pass filter is given by
\begin{align*}
    w_c = \frac{\pi}{M}
\end{align*}
where $f_s$ is the sampling frequency of the signal and $M$ is the decimation factor.

Since the signal is being decimated by a factor of M, the sampling frequency of the signal is also decreased by a factor of M.
Thus, the equation.

\section*{Matlab Code}
Code for the LPF is given below.
\lstinputlisting[language=Matlab]{../code/LPF.m}

Code for the downSampler is given below.
\lstinputlisting[language=Matlab]{../code/downSample.m}

Code for the upSampler is given below.
\lstinputlisting[language=Matlab]{../code/upSample.m}

Code for Convolution is given below.
\lstinputlisting[language=Matlab]{../code/convol.m}

Code for Interpolation and Decimation is given below.
\lstinputlisting[language=Matlab]{../code/interDecim.m}

\section*{Observations}
\begin{itemize}
    \item The mean of absolute error between the original signal and the decimated signal is \textbf{0.0044}.
    \item This shows that the decimated signal is an almost perfect representation of the original signal.
    \item The interpolated signal has a higher sampling rate than the original signal.
    \item The decimated signal has a lower sampling rate than the original signal.
    \item The Low Pass Filter used for interpolation and decimation removes the high frequency components introduced by the zero padding and the decimation respectively.
    \item Gain in the LPF compensates for the loss of energy due to zero padding.
\end{itemize}

\section*{Conclusion}
Interpolation increases signal sampling rates for improved fidelity, while decimation reduces data rates to conserve bandwidth.
Even though the mean absolute error between the original signal and 
the decimated signal is small, the decimated signal is an almost perfect 
representation of the original signal.
They are used for enhancing audio quality to optimizing data transmission in wireless 
networks, emphasizing the importance of these techniques in modern signal processing.

\end{document}